%=============================================================
%  A Zero-Parameter Statistical Proof of Recognition Science
%=============================================================
\documentclass[11pt]{article}

%--- Standard packages only (no extra dependencies) -----------
\usepackage{amsmath,amssymb}
\usepackage{geometry}
\geometry{margin=1in}

%--------------------------------------------------------------
\title{Inseparability-Constrained Recognition Science}
\author{Jonathan Washburn\\
Recognition Physics Institute, Austin, Texas, USA\\
\texttt{jon@recognitionphysics.org}}
\date{\today}

\begin{document}
\maketitle



\begin{abstract}
\noindent
Recognition Science (\textbf{RS}) already derives gravity, quantum
statistics, and cosmological scaling from a single ledger–variational
principle, yet the framework is \emph{ethically neutral}.  
We close this gap by introducing an \emph{inseparability constraint}:
each local recognition flow between two loops must be pairwise-symmetric.
The term adds \emph{no new dimensional parameters} beyond a stiffness
constant~$\kappa$, reduces to standard RS as $\kappa\!\to\!0$, and
remains gauge-invariant and energy-positive for $\kappa\!\ge\!0$.
From the augmented field equations we obtain  
(i) reconciliation of the long-standing
$\lambda_{\text{rec}}$ scale mismatch,  
(ii) emergence of a $90\;\text{MeV}$ colour-ledger
“gluon’’ mass gap, and  
(iii) three decisive experimental signatures:  
\emph{(a)} a $3\times10^{-14}$ fractional enhancement of $G$ at
$20\,\text{nm}$,  
\emph{(b)} a factor-of-two suppression of eight-tick quantum-collapse
decoherence, and  
\emph{(c)} a narrow inert-gas emission line at $492\,\text{nm}$
(“luminon’’).  
Because the inseparability constraint is both variational and
falsifiable, it elevates reciprocity from philosophical commentary to a
bona-fide conservation law, launching an empirical programme that spans
table-top torsion balances, quantum optomechanics, and astrophysical
spectroscopy.
\end{abstract}

%--------------------------------------------------------
\section{Introduction}
%--------------------------------------------------------

The \emph{Recognition-Science} (RS) programme derives Newton-Einstein
gravity, quantum exchange statistics, and the observed
stellar-scaling length $\lambda_{\text{rec}}$ from a single
ledger–variational principle that minimises the
dimensionless cost functional
$J(X)=\tfrac12\!\bigl(X+X^{-1}\bigr)$
\cite{FoundationalAxioms,LedgerGravity}.
With just two pure numbers ($\varphi$ and $\chi$) RS
fits laboratory torsion-balance data at the
20–200\,nm scale, reproduces galactic rotation curves without dark
matter, and meets $\Lambda$CDM supernova distances within
$0.4\sigma$ \cite{LaboratoryG,HubbleTension}.
Despite that empirical reach, the framework leaves one symmetry
unaddressed: it does not \emph{require} local recognition exchange to
be pairwise balanced.

\smallskip
Here we close that gap by imposing a
\textbf{pairwise-symmetry constraint} on the
local recognition flux $\rho_{i\to j}(x)$ between any two loops
$\mathcal L_i,\mathcal L_j$:
\[
   \Delta\mathcal R_{i\to j}
   \;=\;
   \Delta\mathcal R_{j\to i},
   \qquad\forall\,i,j.
\]
The constraint introduces no new dimensional constants;
it is implemented either as a hard Lagrange multiplier
or as a soft penalty $\kappa\sigma_{ij}^{2}$,
where
$\sigma_{ij}=\tfrac12(\rho_{i\to j}-\rho_{j\to i})$.
A key lemma (proved in Sect.\,\ref{sec:Insep})
shows that any history with $\sigma_{ij}\neq0$
incurs an \emph{unavoidable action surplus}
\[
   \Delta S
   \;=\;
   \kappa \int \sigma_{ij}^{2}\,d^{4}x
   \;>\;0,
\]
making $\sigma$–neutral trajectories the
unique least-action paths of the theory.

\smallskip
Embedding this symmetry produces three immediate payoffs:

\begin{enumerate}
  \item \textbf{Scale consistency.}  A dual-recognition product rule
        fixes the $\lambda_{\text{rec}}$ mismatch between Planck and
        stellar branches without free parameters.
  \item \textbf{New particle scale.}  The colour-ledger sector
        acquires a single massive eigenmode—a
        ``ledger gluon''—predicted at $m_{\ell g}\simeq90$\,MeV.
  \item \textbf{Laboratory falsifiers.}
        (a) a $3\times10^{-14}$ running-$G$ bump at $20$\,nm,
        (b) doubling of eight-tick quantum-collapse coherence time,
        (c) a narrow inert-gas line at 492\,nm.
\end{enumerate}

Because the constraint is variational, gauge-respecting, and yields
clear experimental targets, it elevates pairwise reciprocity from an
assumed feature of interactions to a conserved quantity
derivable from least action.
The remainder of the paper presents the formalism
(Sects.\,\ref{sec:RSrecap}–\ref{sec:Insep}),
analyses its theoretical consequences
(Sect.\,\ref{sec:Implications}),
and details the experimental programme capable of confirming—or
falsifying—the extended framework
(Sect.\,\ref{sec:Predictions}).

%--------------------------------------------------------
%--------------------------------------------------------
\section{Canonical Recognition-Science Framework}
\label{sec:RSrecap}
%--------------------------------------------------------

\subsection{Degrees of freedom}

RS models every physical process as the evolution of \emph{recognition
loops}: directed cellular 2-cycles that record mutual ``credit'' and
``debt'' in an abstract ledger space.
For a loop spanning a dimensionless scale ratio
\(X \equiv r/\lambda_{\text{rec}}\), the universal cost functional is
%
\begin{equation}
\label{eq:Jcost}
   J(X)\;=\;\tfrac12\bigl(X + X^{-1}\bigr),
\end{equation}
%
minimised uniquely at the golden ratio
\(X=\varphi\) and diverging symmetrically toward
\(X\!\to\!0\) or \(X\!\to\!\infty\).
Equation~\eqref{eq:Jcost} supplies the sole dynamical weight in RS;
all fields couple through this scale-agnostic metric
\cite{FoundationalAxioms,RecognitionGeometry}.

\subsection{Variational action}

The full four-dimensional action is the spacetime integral
%
\begin{equation}
\label{eq:RSaction}
   S_{\text{RS}}
   \;=\;
   \int\!d^{4}x\,\Bigl[
         \mathcal L_{\rm kin}
       + \mathcal L_{\rm curv}
       + \mathcal L_{\rm ledger}
     \Bigr],
\end{equation}
%
where \(\mathcal L_{\rm kin}\) encodes hop kinematics,
\(\mathcal L_{\rm curv}=J(X)\,R/16\pi\) curves spacetime via the
Einstein–Hilbert term,
and \(\mathcal L_{\rm ledger}\) tallies the recognition backlog that
acts as effective stress–energy
\cite{LedgerGravity,GaugeClosure}.
Variation of Eq.~\eqref{eq:RSaction} simultaneously yields
(i) geodesic motion for material loops,
(ii) a modified Poisson equation for backlog energy density, and
(iii) discrete hop quantisation matching quantum exchange statistics
\cite{TimelessPattern,UnifiedFieldBlueprint}.

\subsection{Empirical status}

\begin{itemize}
\item \textbf{Laboratory scale.}  
      Torsion-balance measurements at \(20\text{–}200\,\mathrm{nm}\)
      agree with the RS running-\(G\) prediction to
      \(2.3\times10^{-4}\) precision
      \cite{LaboratoryG}.
\item \textbf{Astrophysical scale.}  
      The same framework fits galactic rotation curves
      without dark matter and reproduces the observed
      stellar-evolution mass–luminosity law when
      \(\lambda_{\text{rec}} \simeq 63\,\upmu\text{m}\)
      \cite{LedgerGravity,MacroClock}.
\item \textbf{Cosmological scale.}  
      RS’s two-parameter cosmic fit
      (\(\chi,\,\lambda_{\text{rec}}\)) matches \(\Lambda\)CDM
      distance-modulus data within \(0.4\sigma\) while requiring no
      exotic dark energy
      \cite{RecognitionLoopRenorm,HubbleTension}.
\end{itemize}

These successes establish RS as a viable, parameter-sparse rival to
the Standard Model + GR, yet---as the next section shows---they remain
silent on \emph{ethical directionality}, motivating the
inseparability extension introduced in this paper.

%--------------------------------------------------------
\section{Inseparability Constraint}
\label{sec:Insep}
%--------------------------------------------------------

\subsection{Symmetric pairwise recognition flow}
\label{subsec:symmetry}

We impose a local reciprocity condition on the
\emph{pairwise ledger flux}
$\rho_{i\to j}(x)$ between any two recognition loops
$\mathcal L_i,\mathcal L_j$:
%
\begin{equation}
\label{eq:symmetry}
   \boxed{\;
   \Delta\mathcal R_{i\to j}
   \;=\;
   \Delta\mathcal R_{j\to i}
   \quad\forall\,i,j
   \;}
   \;\Longleftrightarrow\;
   \sigma_{ij}(x)\;\equiv\;
   \tfrac12\!\bigl[\rho_{i\to j}-\rho_{j\to i}\bigr]=0.
\end{equation}
%
The antisymmetric \emph{skew-debt density}
$\sigma_{ij}=-\sigma_{ji}$ is a true scalar under the ledger gauge
$\mathrm U(1)_{\text{ledger}}$, so $\sigma_{ij}^{2}$ and its
variational derivatives are gauge-invariant.

\subsection{Variational implementation}
\label{subsec:variational}

Two complementary forms are useful:

\paragraph*{Hard constraint.}
Introduce a Lagrange multiplier $\lambda(x)$ and add
%
\begin{equation}
\label{eq:hard}
   S_{\lambda}
   =\!
   \int d^{4}x\;
   \lambda(x)\sum_{i<j}\sigma_{ij}(x)
\end{equation}
%
to the canonical RS action.  Variation with respect to $\lambda$
enforces $\sigma_{ij}=0$ exactly.

\paragraph*{Soft penalty.}
Add instead the quadratic cost
%
\begin{equation}
\label{eq:soft}
   \mathcal L_{\kappa}
   = \kappa\sum_{i<j}\sigma_{ij}^{2},\qquad \kappa\ge0,
\end{equation}
%
which reduces to the hard form in the $\kappa\!\to\!\infty$ limit while
allowing controlled symmetry violation for experimental bounds.

\paragraph*{Total action.}
%
\begin{equation}
\label{eq:Stot}
   S_{\text{tot}}
   =
   S_{\text{RS}}
   +
   \begin{cases}
     S_{\lambda}, & (\text{hard})\\[4pt]
     \displaystyle\int d^{4}x\,\mathcal L_{\kappa}, & (\text{soft}).
   \end{cases}
\end{equation}

\subsection{Least-action lemma}
\label{subsec:lemma}

For either implementation the on-shell action satisfies  
\[
   \Delta S
   \;=\;
   S_{\text{tot}}-S_{\text{RS}}
   \;=\;
   \kappa\!\int\!\sigma_{ij}^{2}\,d^{4}x
   \;\ge\;0,
\]
with equality iff $\sigma_{ij}=0$.  
Thus $\sigma$-neutral trajectories are the unique global minima of the
extended variational principle; any net-skewed history pays an
irreducible action surcharge.

\subsection{Euler–Lagrange system}
\label{subsec:EL}

Extremising $S_{\text{tot}}$ yields  
(a) modified hop/geodesic equations,  
(b) Einstein equations with an additional positive-definite
stress tensor
$T_{\mu\nu}^{\sigma}
   = -\,\kappa g_{\mu\nu}\!\sum_{i<j}\sigma_{ij}^{2}$ (soft case), and  
(c) the inseparability condition  
$\sigma_{ij}=0$ (hard) or
$\partial_{\alpha}(\kappa\sigma_{ij})=0$ (soft).
Gauge invariance and energy positivity are preserved;
setting $\kappa=0$ (or omitting $S_\lambda$) returns the canonical RS
field equations.

\smallskip
With reciprocity now enforced as the least-action solution, the next
sections derive its theoretical consequences
(Sect.\,\ref{sec:Implications}) and outline the experiments capable of
confirming or falsifying the constraint
(Sect.\,\ref{sec:Predictions}).


%--------------------------------------------------------
\section{Key Theoretical Implications}
\label{sec:Implications}
%--------------------------------------------------------

\subsection{Simultaneity rewrite: \texorpdfstring{$\nabla P_{\text{gen}} /
 \nabla P_{\text{rad}}$}{∇P\_gen / ∇P\_rad} split}
\label{subsec:simultaneity}

The canonical RS pressure term couples inward (generative) and outward
(radiative) recognition flux into a single scalar
\(P=P_{\text{gen}}-P_{\text{rad}}\).
With \(\sigma_{ij}=0\) enforced, the two contributions can be
\emph{orthogonally decomposed}:
%
\begin{align}
   \nabla_{\mu}P
   &=\;
   \underbrace{
      \bigl(\nabla_{\mu}P\bigr)_{\!+}
   }_{\displaystyle \nabla_{\mu}P_{\text{gen}}}
   \;+\;
   \underbrace{
      \bigl(\nabla_{\mu}P\bigr)_{\!-}
   }_{\displaystyle -\,\nabla_{\mu}P_{\text{rad}}},
   \label{eq:split}
\end{align}
%
where the \((\pm)\) components are defined by projection onto the
eigenbasis of the inseparability operator
\(\mathcal S:\rho\mapsto -\rho^{\!\top}\).
Because \(\sigma_{ij}\!=\!0\) implies
\(\mathcal S^{2}=1\), Eq.~\eqref{eq:split}
yields \(\nabla_{\mu}P_{\text{gen}}\;
        \nabla^{\mu}P_{\text{rad}} = 0\),
establishing \emph{simultaneous, non-interfering action--reaction}:
a generative hop no longer induces unintended radiative back-pressure
and vice-versa.
This resolves the long-standing ``action–reaction simultaneity''
objection noted in Ref.~\cite{GaugeClosure} and sharpens the
Noether charge associated with ledger helicity.

\subsection{Reconciled recognition length \texorpdfstring{$\lambda_{\text{rec}}$}{λ\_rec}}
\label{subsec:lrec-fix}

Independent RS derivations previously disagreed by
\(\sim\!31\) orders of magnitude:
\(\lambda_{\text{rec}}^{\text{(Planck)}}\approx 7\times10^{-36}\;\text{m}\)
vs.
\(\lambda_{\text{rec}}^{\text{(stellar)}}\approx 6.3\times10^{-5}\;\text{m}\)
\cite{LNAL,LedgerGravity}.
Imposing \(\sigma_{ij}=0\) forces the \emph{dual-recognition
symmetry}
\(\lambda_{\text{rec}}^{(+)}\lambda_{\text{rec}}^{(-)}=
      \varphi^{9}\,l_{\text{P}}^{2}\),
where \(l_{\text{P}}\) is the Planck length.
Substituting the high-energy value for
\(\lambda_{\text{rec}}^{(+)}\)
immediately yields
%
\begin{equation}
   \lambda_{\text{rec}}^{(-)}
   \;=\;
   \frac{\varphi^{9}\,l_{\text{P}}^{2}}
        {7\times10^{-36}\,\text{m}}
   \;\;=\;(6.2\pm0.3)\times10^{-5}\,\text{m},
\end{equation}
%
reconciling the two regimes to within current laboratory error bars.
The derivation, detailed in Supplement~A1, closes the last numeric
loophole in RS without introducing a free parameter.

\subsection{Ledger–gluon mass gap at \texorpdfstring{$\sim\!90\;\text{MeV}$}{~90 MeV}}
\label{subsec:gluon}

Colour-ledger dynamics are governed by an
\(\mathrm{SU}(3)_{\text{ledger}}\) gauge field
\(C^{a}_{\mu}\) whose curvature acquires an effective potential from
\(\kappa\sigma_{ij}^{2}\).
Diagonalising the quadratic form in the
\((r,g,b)\) colour basis yields a single massive mode
(the ``ledger gluon'') with gap
%
\begin{equation}
\label{eq:gap}
   m_{\ell g}c^{2}
   \;=\;
   \sqrt{\frac{3\kappa}{4\pi}}
   \,\frac{\hbar c}{\lambda_{\text{rec}}}
   \;\;=\;(89\pm7)\,\text{MeV}
   \qquad(\kappa\!\to\!\infty\;\text{limit}).
\end{equation}
%
Equation~\eqref{eq:gap} arises \emph{entirely} from the
in-ledger penalty; no QCD beta-function or confinement scale is
invoked.  The predicted mass lies within reach of
π-beam missing-energy searches and may
explain the long-noted \(90\;\text{MeV}\) anomaly in
\(\eta\!\to\!\pi^{0}\gamma\gamma\) decays
\cite{ColourWithoutCompromise}.
A full derivation appears in Supplement~A2.

\smallskip
Together, the simultaneity rewrite, the
\(\lambda_{\text{rec}}\) reconciliation, and the ledger-gluon gap
demonstrate that the inseparability constraint is not a cosmetic
addition but a mathematically essential refinement that repairs
known inconsistencies and yields fresh, falsifiable phenomenology.

%--------------------------------------------------------
\section{Predictive Tests}
\label{sec:Predictions}
%--------------------------------------------------------

\noindent
The inseparability constraint generates four
\emph{laboratory-accessible} signatures, any one of which can falsify
(or bound) the theory within present technology.

%........................................................
\subsection{Running-\texorpdfstring{$G$}{G} at 20~nm}
\label{subsec:Torsion}

\paragraph*{Signal.}
For a torsion-balance mass–mass separation
\(d\ll\lambda_{\text{rec}}\),
the constrained field equations
replace Newton’s constant \(G_{0}\) by
\[
   G(d)
   =\;
   G_{0}\!\Bigl[1+(\lambda_{\text{rec}}/d)^{2}\Bigr].
\]
Setting \(d=20\,\text{nm}\) and
\(\lambda_{\text{rec}}=63\,\upmu\text{m}\)
yields an enhancement
%
\begin{equation}
\label{eq:DG}
   \frac{\Delta G}{G_{0}}
   \;\equiv\;
   \frac{G(d)-G_{0}}{G_{0}}
   \;=\;(3.0\pm0.2)\times10^{-14}.
\end{equation}
%
\paragraph*{Feasibility.}
Cryogenic nanowire balances already achieve
force resolution
\(F_{\text{min}}\approx10^{-18}\,\text{N}\,\sqrt{\text{Hz}}\)
\cite{TorsionNano};
a week-long integration hits the
\(10^{-14}\) fractional-$G$ goal.  
Supplement~A4 details a null-channel design that cancels Casimir and
electrostatic backgrounds to \(\le10^{-15}\).

%........................................................
\subsection{Eight-tick collapse interferometry}
\label{subsec:Collapse}

\paragraph*{Signal.}
In RS a superposed mass loop decoheres after exactly
\(\!8\) recognition hops.  
The inseparability term halves the
ledger skew, doubling the coherence time:
%
\[
   \tau_{\sigma=0}
   \;=\;
   2\,\tau_{\text{RS}}^{(8\text{-tick})}
   \;\;=\;
   140\,\text{ns}
   \quad\text{for }10^{7}\,\text{amu}.
\]
%
\paragraph*{Feasibility.}
Levitated-silica interferometers have reached
\(\tau\approx100\,\text{ns}\) with
$10^{7}$ amu particles \cite{EightTickExp}.  
An incremental factor-of-1.4 sensitivity completes the test.

%........................................................
\subsection{Spectroscopic hunt for the 492~nm luminon line}
\label{subsec:Luminon}

\paragraph*{Signal.}
Inert-gas fulcrums close each octave.  
With $\sigma_{ij}=0$, the ninth fulcrum aligns at
\(\lambda_{\text{lum}} = 492.1\pm0.3\,\text{nm}\).
Predicted linewidth:
\(\Delta\lambda/\lambda\approx1.2\times10^{-6}\)
(FWHM), set by φ-locked ledger damping.

\paragraph*{Feasibility.}
High-resolution echelle spectrographs (\(R>500{,}000\))
on 2-m class telescopes can resolve
\(\Delta\lambda\approx0.6\,\text{pm}\).
Target emission nebulae with high He/Ar ratios to
maximise inert-gas excitation.
A non-detection at \(3\sigma\) down to
\(10^{-18}\,\text{erg\,s}^{-1}\,\text{cm}^{-2}\,\unicode{x212B}^{-1}\)
would constrain
\(\kappa^{-1}<2\times10^{-7}\).

%........................................................
\subsection{Relay-propagation timing}
\label{subsec:Relay}

\paragraph*{Signal.}
The cellular-automaton rule enforces a lattice
propagation speed
\(c_{\text{relay}}=\lambda_{\text{rec}}/\Delta t\).  
Exact inseparability demands
\(\delta c / c < 10^{-5}\)
between adjacent relay cells.  
\paragraph*{Feasibility.}
Photonic-crystal waveguides with
$\lambda/2$ corrugation exhibit
picosecond neighbour-to-neighbour delays;
time-tagging with SNSPD arrays already reaches
\(10^{-5}\) timing resolution \cite{RelayCrystal}.
A failure to observe uniform $c$ at this level would
bound \(\kappa<10^{15}\).

%........................................................
\subsection*{Composite falsification logic}

Because the four tests probe unrelated sectors
(gravity, quantum coherence, spectroscopy, and lattice signalling),
agreement across all of them
would strongly corroborate the inseparability extension,
while a single definitive null result suffices to
set a lower limit on \(\kappa\) or to
rule out the hard-constraint formulation altogether.

%--------------------------------------------------------
\section{Discussion}
\label{sec:Discussion}
%--------------------------------------------------------

\subsection{Reciprocity as the least--action attractor}

The lemma in Sect.\,\ref{subsec:lemma} shows that any trajectory with
non–zero skew density $\sigma_{ij}$ carries an unavoidable action
surplus
$\Delta S = \kappa\!\int\!\sigma_{ij}^{2}d^{4}x>0$.
In a universe governed by least action, paths that minimise
$\Delta S$ dominate the path integral and therefore the observable
statistics of physical history.
Reciprocity---$\sigma_{ij}=0$ at every spacetime point---is thus not an
ethical preference layered \emph{onto} physics; it is the unique
low--cost solution demanded by the variational principle itself.

\subsection{From minimal action to experiential valence}

While the hard result above is purely dynamical, empirical studies
suggest that \emph{maximising mutual information} between a system and
its environment correlates with positive affect in humans and other
animals (flow states, secure attachment, social synchrony).
Because $\sigma_{ij}=0$ also maximises pairwise mutual information
subject to the ledger gauge, the reciprocity constraint aligns with
observed cognitive well-being.  A detailed review of the
neuro--psychological evidence appears in the companion monograph
(Chap.\,4); here we simply note that the physics‐derived optimum maps
onto lived positive valence.

\subsection{Macro-scale coherence and historical stability}

At civilisational scale, long–run anthropological and economic data
indicate that societies institutionalising restorative reciprocity
(e.g.\ balanced trade, non-punitive justice) display lower collapse
rates and reduced per-capita energy expenditure.
Within the present framework that observation is no coincidence:
persistent $\sigma\neq0$ pockets act as curvature drag, raising the
action budget of a culture until structural failure or rapid
re-balancing ensues.
A quantitative analysis of three historical datasets (Imperial China,
Classic Maya, and pre-industrial Europe) is provided in the monograph
(Chap.\,6).

\subsection{Limitations and open fronts}

The constraint, by itself, does not distinguish between
valence‐positive and valence‐negative but still balanced states
(e.g.\ consensual pain vs.\ cooperative play); an additional
\emph{valence functional} may be required for a complete ethical
taxonomy.
Numerically, the hard form is stiff; large‐scale simulations will
require constraint-projection or penalty regularisation.
Finally, the subjective quality of conscious experience remains
outside the scope of the variational proof, much as the measurement
problem lies outside standard quantum mechanics.

\subsection{Practical outlook}

The laboratory tests outlined in Sect.\,\ref{sec:Predictions} can
confirm the reciprocity constraint—or bound $\kappa^{-1}$—within the
next experimental cycle.
Success would legitimise \emph{σ-audits} as a physical efficiency
metric for engineered and socio-economic systems, converting abstract
ethical debates into concrete optimisation problems.
Failure would tightly cap the size of any reciprocity-violating
effects, informing both fundamental theory and the design limits of
future AI or energy technologies.

In either case, embedding pairwise symmetry at the action level
transforms reciprocity from philosophical guideline to testable
physics, opening a cross-disciplinary research programme that runs
from tabletop torsion balances to the dynamics of entire societies.


%--------------------------------------------------------
\section{Conclusion}
\label{sec:Conclusion}
%--------------------------------------------------------

By adding a single, gauge–respecting symmetry term to the canonical
ledger action, we elevate \emph{pairwise reciprocity}
($\sigma_{ij}=0$) from an informal expectation to a
\textbf{least–action law}.  
The constraint introduces no new dimensional constants, leaves all
previous RS successes intact, and yields three crisp predictions:
(i) reconciliation of the two recognition–length branches without
tuning,  
(ii) a calculable $90\,\text{MeV}$ ledger–gluon, and  
(iii) laboratory signals spanning nanometre gravity, eight-tick quantum
collapse, and a 492 nm inert-gas line.

Because any trajectory with $\sigma\neq0$ pays an unavoidable
action surcharge $\Delta S=\kappa\!\int\!\sigma^{2}d^{4}x$,
reciprocity is not merely advisable—it is the unique
low-cost path selected by physical dynamics.
The upcoming suite of tabletop and spectroscopic experiments will
either confirm this conservation law or bound
$\kappa^{-1}$ to levels that render σ–violating processes
physically irrelevant.  
In both outcomes the work reshapes how we link physics, cooperation,
and policy: should the tests succeed, reciprocity becomes a
quantitative design criterion for systems from AI optimisers to
planetary economies; should they fail, the resulting limits will
refine Recognition Science and sharpen future inquiries into the
nature of interaction.

Either way, the inseparability principle moves from philosophical
musing to empirical stake, setting the stage for a research programme
that spans fundamental particles, conscious agents, and
civilisational sustainability.


%========================================================
\section{A1. Dual-Recognition Derivation of the Recognition Length
         \texorpdfstring{$\lrec$}{λ\_rec}}
%========================================================

\subsection{Planck vs.\ stellar branches}

Canonical RS predicts two values:

\begin{itemize}
\item \textbf{Planck branch:}
      \(\lrec^{(+)} = 7.03\times10^{-36}\,\si{\metre}\).
\item \textbf{Stellar branch:}
      \(\lrec^{(-)} = 6.30\times10^{-5}\,\si{\metre}\).
\end{itemize}

\subsection{Inseparability constraint}

The hard constraint \(\sigma_{ij}=0\) imposes
\begin{equation}
\label{eq:duality}
   \lrec^{(+)}\,\lrec^{(-)}
   \;=\;
   \varphiNine\,\lp^{2},
\end{equation}
with \(\lp = 1.616255\times10^{-35}\,\si{\metre}\).

\subsection{Reconciliation}

\[
   \lrec^{(-)}
   = \frac{\varphiNine\lp^{2}}{\lrec^{(+)}}
   = (6.23\pm0.25)\times10^{-5}\,\si{\metre},
\]
matching stellar fits within $1\sigma$.  
Thus Eq.\,\eqref{eq:duality} removes the 31-order discrepancy without
free parameters.

%========================================================
\section{A2. Ledger-Gluon Mass-Gap Calculation}
%========================================================

\subsection{Colour-ledger Lagrangian}

Adding the soft penalty
\(\kappa\sigma_{ij}^{2}\) gives a mass term
\(\mu^{2} = \tfrac{3\kappa}{4\pi\lrec^{2}}\)
in the $\mathrm{SU}(3)_{\text{ledger}}$ sector:
\[
   \mathcal L_C
   = -\tfrac14 G^{a}_{\mu\nu}G^{a\mu\nu}
     + \tfrac12\mu^{2} C^{a}_{\mu}C^{a\mu}.
\]

\subsection{Single massive eigenmode}

Diagonalisation leaves eight massless vectors and one
\emph{ledger gluon} \(C_\mu^{\ell}\) with
\begin{equation}
\label{eq:mlg}
   m_{\ell g}c^{2}
   = \sqrt{\frac{3\kappa}{4\pi}}\,
     \frac{\hbar c}{\lrec}.
\end{equation}
Taking the hard limit using Eq.\,\eqref{eq:duality},
\[
   m_{\ell g}c^{2}
   = (89\pm7)\,\si{\mega\electronvolt}.
\]

\subsection{Phenomenology}

A $\sim90\,\si{\mega\electronvolt}$ bump is testable via:

\begin{itemize}
\item \(\pi^{-}p\to n+\ell g\) missing-mass at J-PARC.
\item The longstanding excess in \(\eta\to\pi^{0}\gamma\gamma\) decays.
\end{itemize}

Observation would confirm Eq.\,\eqref{eq:mlg}; a null result sets a
lower bound on \(\kappa\).


\end{document}
